\documentclass[11pt]{article}
\usepackage[letterpaper, margin=1.25in]{geometry}
\usepackage{longtable,tabu}
\usepackage{graphicx}
\usepackage{hyperref}

\begin{document}
\begin{titlepage}
   \vspace*{\stretch{1.0}}
   \begin{center}
      \Large\textbf{Run Tank Stand User Manual}\\
      \bigskip
      \large\textit{Robin Liu - Mech 2018}\\
      \bigskip
      Feb 2018
   \end{center}
   \vspace*{\stretch{2.0}}
\end{titlepage}

\section{Scope}
This document does not provide the engineering analysis behind the design, but instead just provides some insight into how to use the thing. If you are interested in a somewhat outdated analysis, search on the drive for my work term report that I wrote on the subject.

This document \textit{does} discuss the water jacket and load cell.

\section{Consumables/Replaceables Quick Reference Guide}
\begin{enumerate}
	\item O Rings are -360 and -366 size. You need at least one of each.
	\item All bolts that touch the vertical columns (except the bottom four) are 1/4-20, and should be 1 long. Slightly longer or shorter is fine, but 1" is recommended. For your sanity they should be fully threaded. By my count around 50 are required. Accordingly, about 100 washers and 50 nuts are needed.
	\item The bolts connecting the base plate to the legs are 3/8-16. They need to be at least 2 inches long, plus or minus 0.25 (longer is ok, but becomes progressively more annoying). The partially threaded 2" bolts with 1" threaded are good for this.
	\item The bolts connecting the columns to the base plate are also 3/8-16. They should be about 1.5 inches long.
	\item Screws for the feet are \#10-24, and at least 2" long
\end{enumerate}

\section{Full Bill of Materials}
\begin{center}
	\begin{longtabu} to \linewidth { |X[l]|X[2,l]|X[l]|X[0.7,r]| }
		\hline
		\textbf{Item Name} & \textbf{Item Description} & \textbf{Suggested Source} & \textbf{Quantity} \\
		\hline
		Columns & 1.5"x1.5"x1/8" steel angle, 7.5 feet in length & E3 & 4 \\
		\hline
		Top plate & 1'x1'x1/8" steel plate & E3 & 1 \\
		\hline
		Base plate & 13"x24"x1/4" steel plate & E3 & 1 \\
		\hline
		Legs & 1"x1"x1/8" square tube & E3 & 2 \\
		\hline
		Column support brackets & 1.5"x1.5"x1/8" steel angle, 1.5" long & E3 & 4 \\
		\hline
		Column stiffening plates & 12"x3"x1/8" steel plate & E3 & 8 \\
		\hline
		Top plate brackets & 1.5"x1.5"x1/8" steel angle, 1.25" long & E3 & 4 \\
		\hline
		Feet & Rubber bumper with unthreaded hole & McMaster-Carr\footnote{\url{https://www.mcmaster.com/\#9540k793/=1as9gh9}} & 4 \\
		\hline
		Shield panels & 12"x48"x1/4" polycarbonate sheet & E3 & 4\\
		\hline
		Feet fasteners & \#10-24, min 2" long, with washers and nuts	 & E5 & 4 \\
		\hline
		Leg fasteners & 3/8-16, min 2" long, with washers and nuts & Wherever & 4 \\
		\hline
		All other fasteners & 1/4-20, min 1" (ideally 1.5"), with washers and nuts & Wherever & 56 (112 washers) \\
		\hline
		Water jacket sealing caps & Machined aluminum caps to seal between the water jacket shell and the run tank & Make it ourselves & 1 \\
		\hline
		Water jacket shell & 6" OD/ 7.75" ID polycarbonate tube, 36" length & McMaster-Carr\footnote{\url{https://www.mcmaster.com/\#8585k59/=1aswn3e}} & 1 \\
		\hline
		Water jacket inlet adapter & 3D printed custom shape & 3D Print centre or WatIMake & 1 \\
		\hline
		Water jacket outlet adapter & 3D printed custom shape & 3D Print centre or WatIMake & 1 \\
		\hline
		O Ring - shell side & -366 & McMaster-Carr\footnote{\url{https://www.mcmaster.com/\#9452k391/=1asx6f4}} & 1 \\
		\hline
		O Ring - tank side & -360 & McMaster-Carr\footnote{\url{https://www.mcmaster.com/\#9452k516/=1as82k7}} & 1 \\
		\hline
		Hose clamps & 8" ones for the adapters around the jacket, and minimum 1.5" ones for the hose-to-adapter connection & Home Depot, McMaster-Carr\footnote{\url{https://www.mcmaster.com/\#5011t43/=1atedgj}} & 4 large, 2 small \\
		\hline
	\end{longtabu}
\end{center}

Some items are not mentioned in the table, such as the entire run tank assembly (for obvious reasons), the inlet and outlet hoses (mostly because I don't have those dimensions on hand and you can easily replace them with pretty much anything as long as it's flexible enough and has the same size), and the load cell with associated material (it's pretty self-explanatory, just hang it by a threaded rod using nuts and stuff).

\section{Assembly Reference}
\subsection{Stand Assembly}
As much as possible, the stand should be left in an assembled or mostly assembled state. There's a lot of stuff to align so it's best if it's not moved around too much. The order of assembly for the stand is roughly detailed:
\begin{enumerate}
	\item setup all the columns with shields and stiffener plates on 3/4 sides (try to datum the bottom off of a flat surface while doing so). These should go on the inside of the columns
	\item Attach the feet to the legs
	\item Attach the legs to the base plate
	\item Attach the columns to the base plate using the brackets. Fiddle with it until it stands up straight and the columns are all touching the base plate instead of being held up by other columns and/or the bracket.
	\item Attach the top plate brackets to the columns. Fiddle with positioning until the top surface of all the top plate brackets are at the same elevation and nicely flat. Make sure that the brackets are attached to the outside of the columns, folding over and inwards (to fold over the columns, so to speak)
	\item Attach the top plate to the brackets, on top.
	\item Hang the load cell, checking for:
		\begin{itemize}
			\item Proper engagement length into the load cell, so it doesn't accidentally fall off the threaded rod
			\item Double nut and double washer hanging the threaded rod from the top plate
			\item A nice amount of extra threaded rod length sticking out the top (probably at least 2")
			\item Some method of preventing load cell from unthreading itself (a nut tightened against it is good, threadlocker is also good)
			\item Proper tank height so that the blast shields are effective
		\end{itemize}
	\item After the tank is hung with all the hose attachments done, affix the last blast shield panel. This panel can go on the outside with only the 4 corner connections. \textbf{This side needs to face away from the viewing area.}
\end{enumerate}

\subsection{Water Jacket Assembly}

\begin{enumerate}
	\item Of special note is that the water jacket (at least the sealing cap) needs to be assembled alongside the oxidizer tank, as it's held in place by the bolt circles.
	\item Put O rings into the sealing cap
	\item Affix inlet and outlet adapters (don't forget to apply gasket and/or other sealing strategies)
	\item Slide the sealing cap onto the bottom end of the oxidizer tank. Check for ripped O-Rings
	\item Assemble the remainder of the oxidizer tank.
	\item Slide the jacket shell on from the top end of oxidizer tank (keeping the inlet on the \textbf{\textit{bottom}}.
	\item Use some duct tape to hold the sealing caps onto the water jacket shell.
	\item The top of the water jacket can be held concentric with the tank by using shims/wedges.
	\item After the rest of the tank is assembled, and the whole thing is brought outside and is ready to be hooked onto the load cell, attach the hoses to the inlet and outlet adapters. \textbf{Be aware of where the hoses route}. The open side of the tank stand (final panel) must face away from the spectator area. Therefore the hoses should already be routed through the gap at the bottom of the tank stand.
		\begin{itemize}
			\item Note that the hoses definitely require nicely tightened hose clamps and probably some sort of sealing (gasket tape or something like that).
		\end{itemize}
	\item Run a flow test to make sure that water is flowing well. If anything starts spinning (especially the load cell, that is bad news bears and you should rectify the situation immediately.
\end{enumerate}
\end{document}
